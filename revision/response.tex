\documentclass{article}
\usepackage{a4wide}

\usepackage{amsthm}
\usepackage{url}
\newtheorem{Comments}{\textbf{Comments}}


\title{Summary of Changes: \\Temporal Graph Cube}
\author{Guoren~Wang,
	Yue~Zeng,
	Rong-Hua~Li,
	Hongchao~Qin,
	Xuanhua Shi,\\
	Yubin Xia, 
	Xuequn Shang, 
	Liang Hong
}
\date{}

\begin{document}
%\Large
\maketitle

We appreciate the opportunity to revise our work, and are very grateful to the anonymous reviewers for their insightful and valuable comments. We have revised the manuscript and tried our best to address all the comments. 

The main changes are summarized below.
\begin{itemize}
%\item  As suggested by the reviewer, we have added new experiments against alternative solutions in the revised version (see exp-5 of section 4.1). 

\item test

\end{itemize}

The point-to-point responses are given below.
	
\section{Response to Reviewer \#1}
\begin{Comments}
The expression of motivation is not clearly. Does the limitation of static graph cube is it cannot support range queries or it only can support latest data?
\end{Comments}
\noindent \textbf{Response:} [todo]. The major limitation of static graph cube is that it cannot support temporal multidimensional networks. Static graph cube can only be built on static multidimensional networks, e.g., DBLP where temporal information of co-authorships is omitted. Consequently, static graph cube cannot support range queries since there is no temporal information in its dataset. Static graph cube cannot support latest data because its authors didn't consider that new edges may be added into the static networks. Temporal graph cube supports latest data because we will update the temporal networks and indexes when new edges are added.

\begin{Comments}
Authors should improve their writing. Especially the punctuation makes the definitions hard to be understood.
\end{Comments}
\noindent \textbf{Response:} test

\begin{Comments}
Authors provide similarity of snapshots and two ratios in the Sec. 5. All of them are calculated based on the edges’ number. Why do you use similarity but not two ratios to proof the effectiveness?
\end{Comments}
\noindent \textbf{Response:} Actually we do use the two ratios to proof the effectiveness of the index. As in our experiments (see Table 6 of Section 7.3), we show the two ratios in the 5-th and 8-th column to proof the actual effectiveness of the index. Similarity of snapshots is the metric which has a strong correlation with the two ratios, so it is used to measure the necessity of building index for a materialized view. In detail, both theoretically and experimentally we find that larger similarity of snapshots in a materialized view leads to higher acceleration ratio and lower space consumption ratio of the index. If the value of similarity of snapshots in a view is as low as $ 1 $, then there is no need to build index for the view and the baseline method to conduct queries is the best method, as we analyzed in Section 5.
%Acceleration ratio in Definition 7 and space consumption ratio in Definition 9 theoretically measure the query efficiency and space cost of the index respectively based on the edges' number. As in our experiments (see Table 6 of Section 7.3), we show the two ratios in the 5-th and 8-th column to proof the actual effectiveness of the index. However, both theoretically and experimentally we find that the query efficiency and space cost of the index are both related to a metric named 'similarity of snapshots' in the manuscript, as shown in Table 6. If the value of similarity of snapshot. To sum up, we do use the two ratios to proof the effectiveness of the index, and similarity of snapshots is used to 

\begin{Comments}
In the experiments, authors should add an experiment of comparison with existing methods to evaluate the efficiency of their model.
\end{Comments}
\noindent \textbf{Response:} Following your advice, we have added experiments of existing methods in Section 7.2 and Exp-1 of Section 7.3 to evaluate the effectiveness and efficiency of our model. First, in Section 7.2 we add experimental results of queries in static graph cube. By comparing the query results we show that queries in temporal graph cube can reveal hidden information which may be ignored using queries in static graph cube. Second, at the end of Exp-1 of Section 7.3 we implement two other kinds of indexes, prefix array and sparse table, to accelerate range queries of temporal graph cube. These two indexes both can achieve $ O(1) $ time complexity in conducting range queries on normal numeric arrays. However, in the case of snapshot arrays, only sparse table achieves higher acceleration ratio, but with unacceptable space consumption ratio (sometimes out of memory). Overall, segment-tree based method in manuscript is the most suitable method to accelerate range queries in temporal graph cube.

\begin{Comments}
The study of related works is insufficient, and some of the results and limitations of existing works are not discussed.
\end{Comments}
\noindent \textbf{Response:} Thanks for pointing out this problem. We have reviewed more works mainly related to graph OLAP and graph stream summarization in Section 8 of the revised version, and we have added discussion about limitations of all the mentioned existing works.

\begin{Comments}
The second parameter of MergeSnapshotExtre on Line 11, Algorithm 5 should be (u’, v’, t, a)?
\end{Comments}
\noindent \textbf{Response:} The parameter should remain $ (u',v',a,a) $. In line 11 of Algorithm 5, the algorithm inserts $ e' $ into $ treeNode.snapshot $, which is equivalent to merge two snapshots: $ treeNode.snapshot $ and a temporary snapshot consisting of only $ e' $. However, edges in snapshot of each tree node are in form of $ (u,v,max,min) $, so we need to transform $ e' $ into $ (u',v',a,a) $, since $ a $ is both the maximum and minimum value of the numeric attribute in $ e' $. $ t $ is only used to determine whether the left child or the right child of $ treeNode $ should also be updated.


\section{Response to Reviewer \#2}
\setcounter{Comments}{0}
\begin{Comments}
D1: The management of snapshots should be described with more details. According to Def. 2, the snapshot is defined with a single timestamps. It looks like that when an edge forms at timestamp t, it never disappears. How do we maintain an edge with time range from t1 to t2? Otherwise, the limitation or applicability of the proposed method should be discussed.
\end{Comments}
\noindent \textbf{Response:} It is reasonable for a temporal edge to have only one timestamp, because in many cases, durations or end time of temporal edges are meaningless. For example, DBLP only records date of publication of research papers. For a research paper, the end time of publication or existence is meaningless. We just need to focus on the fact that at a certain time, some researchers have published a certain paper. Given a time range $ [t_1,t_2] $, it's easy to determine whether a research paper is published in $ [t_1,t_2] $. It's the same with movies in IMDB. Another example is email communication network, which is also temporal network. Temporal edges in email communication network contain only timestamp of sending the email, and there is no need to store timestamp of finishing sending the email or timestamp of deleting the email. We just need to focus on the fact that there was an email exchange between certain people at a certain time, or we can consider this as an instant activity.

The temporal edges you mentioned do exist in temporal networks, such as phone calls network, since each phone call lasts for a while. However,

\begin{Comments}
D2: Following D1, in current experimental datasets, DBLP and IMDB, all edges with single timestamps. I suggest adding experiments with datasets which contain edges with time ranges.
\end{Comments}
\noindent \textbf{Response:} test

\begin{Comments}
D3: The effectiveness of partial materialization is not verified in experiments.
\end{Comments}
\noindent \textbf{Response:} test

\section{Response to Reviewer \#3}
\setcounter{Comments}{0}
\begin{Comments}
The experimental study only tests the proposed methods in this paper. I know the temporal graph cube is a novel problem studied in this paper, thus there is no existing algorithms/methods can direct handle it. However, is it possible to adapt some existing methods on graph summarizing, or static graph cube methods to handle the temporal graph cube problem as some baselines? Through comparing with modified existing methods can better show the advantages of the proposed methods.
\end{Comments}
\noindent \textbf{Response:} test


\begin{Comments}
 some minor language issues should be addressed. E.g., In line 52, left column, page 2. "a index" $ \rightarrow $ "an index"
\end{Comments}
\noindent \textbf{Response:} Thank you very much for pointing out the negligence. We have corrected this error.


\end{document}


