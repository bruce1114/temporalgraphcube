\documentclass{article}
\usepackage{a4wide}

\usepackage{amsthm}
\usepackage{url}
\newtheorem{Comments}{\textbf{Comments}}


\title{Summary of Changes: \\Temporal Graph Cube}
\author{Guoren~Wang,
	Yue~Zeng,
	Rong-Hua~Li,
	Hongchao~Qin,
	Xuanhua Shi,\\
	Yubin Xia, 
	Xuequn Shang, 
	Liang Hong
}
\date{}

\begin{document}
%\Large
\maketitle

We appreciate the opportunity to revise our work, and are very grateful to the anonymous reviewers for their insightful and valuable comments. We have revised the manuscript and tried our best to address all the comments. 

The main changes are summarized below.
\begin{itemize}
%\item  As suggested by the reviewer, we have added new experiments against alternative solutions in the revised version (see exp-5 of section 4.1). 

\item As suggested, we have added experiments of existing methods in Section 7.2 and Exp-1 of Section 7.3 to evaluate the effectiveness and efficiency of our model.
\item We have analyzed how to easily adapt our method to support temporal networks with temporal edges containing time range at the end of Section 7.1.
\item We have added experiments related to partial materialization strategy in Exp-3 of Section 7.3.
\item We have reviewed more related works and added discussion about limitations of all the mentioned existing works in Section 8.

\end{itemize}

The point-to-point responses are given below.
	
\section{Response to Reviewer \#1}
\begin{Comments}
The expression of motivation is not clearly. Does the limitation of static graph cube is it cannot support range queries or it only can support latest data?
\end{Comments}
\noindent \textbf{Response:} The major limitation of static graph cube is that it cannot support temporal multidimensional networks. Static graph cube can only be built on static multidimensional networks, e.g., DBLP without temporal information of co-authorships. Consequently, static graph cube cannot support range queries since there is no temporal information in its dataset. Static graph cube cannot support latest data because its authors didn't consider that new edges may be added into the static networks. Temporal graph cube supports latest data because we will update the temporal networks and indexes when new edges are added.

\begin{Comments}
Authors should improve their writing. Especially the punctuation makes the definitions hard to be understood.
\end{Comments}
\noindent \textbf{Response:} Following your advice, we have tried to update some definitions by replacing sentences separated by punctuation with relatively complete sentences.

\begin{Comments}
Authors provide similarity of snapshots and two ratios in the Sec. 5. All of them are calculated based on the edges’ number. Why do you use similarity but not two ratios to proof the effectiveness?
\end{Comments}
\noindent \textbf{Response:} Actually we do use the two ratios to proof the effectiveness of the index. As in our experiments (see Table 6 of Section 7.3), we show the two ratios in the 5-th and 8-th column to proof the actual effectiveness of the index. Similarity of snapshots is the metric which has a strong correlation with the two ratios, so it is used to measure the necessity of building index for a materialized view. In detail, both theoretically and experimentally we find that larger similarity of snapshots in a materialized view leads to higher acceleration ratio and lower space consumption ratio of the index. If the value of similarity of snapshots in a view is as low as $ 1 $, then there is no need to build index for the view and the baseline method is the best method, as we analyzed in Section 5.
%Acceleration ratio in Definition 7 and space consumption ratio in Definition 9 theoretically measure the query efficiency and space cost of the index respectively based on the edges' number. As in our experiments (see Table 6 of Section 7.3), we show the two ratios in the 5-th and 8-th column to proof the actual effectiveness of the index. However, both theoretically and experimentally we find that the query efficiency and space cost of the index are both related to a metric named 'similarity of snapshots' in the manuscript, as shown in Table 6. If the value of similarity of snapshot. To sum up, we do use the two ratios to proof the effectiveness of the index, and similarity of snapshots is used to 

\begin{Comments}
In the experiments, authors should add an experiment of comparison with existing methods to evaluate the efficiency of their model.
\end{Comments}
\noindent \textbf{Response:} Following your advice, we have added experiments of existing methods in Section 7.2 and Exp-1 of Section 7.3 to evaluate the effectiveness and efficiency of our model. First, in Section 7.2 we add experimental results of queries in static graph cube. By comparing the query results we show that queries in temporal graph cube can reveal hidden information which may be ignored using queries in static graph cube. Second, at the end of Exp-1 of Section 7.3 we implement two other kinds of indexes, prefix array and sparse table, to accelerate range queries of temporal graph cube. These two indexes both can achieve $ O(1) $ time complexity in conducting range queries on normal numeric arrays. However, in the case of snapshot arrays, only sparse table achieves higher acceleration ratio, but with unacceptable space consumption ratio (sometimes out of memory). Overall, segment-tree based method in manuscript is the most suitable method to accelerate range queries in temporal graph cube.

\begin{Comments}
The study of related works is insufficient, and some of the results and limitations of existing works are not discussed.
\end{Comments}
\noindent \textbf{Response:} Thanks for pointing out this problem. We have reviewed more works mainly related to graph OLAP and graph stream summarization in Section 8 of the revised version, and we have added discussion about limitations of all the mentioned existing works.

\begin{Comments}
The second parameter of MergeSnapshotExtre on Line 11, Algorithm 5 should be (u’, v’, t, a)?
\end{Comments}
\noindent \textbf{Response:} The parameter should remain $ (u',v',a,a) $. In line 11 of Algorithm 5, the algorithm inserts $ e' $ into $ treeNode.snapshot $, which is equivalent to merge two snapshots: $ treeNode.snapshot $ and a temporary snapshot consisting of only $ e' $. However, edges in snapshot of each tree node are in form of $ (u,v,max,min) $, so we need to transform $ e' $ into $ (u',v',a,a) $, since $ a $ is both the maximum and minimum value of the numeric attribute in $ e' $. $ t $ is only used to determine whether the left child or the right child of $ treeNode $ should also be updated.


\section{Response to Reviewer \#2}
\setcounter{Comments}{0}
\begin{Comments}
D1: The management of snapshots should be described with more details. According to Def. 2, the snapshot is defined with a single timestamps. It looks like that when an edge forms at timestamp t, it never disappears. How do we maintain an edge with time range from t1 to t2? Otherwise, the limitation or applicability of the proposed method should be discussed.
\end{Comments}
\noindent \textbf{Response:} Thanks for your advice. We do omitted the case where temporal edges contain time ranges. The motivation of our work is based on the two datasets we found, DBLP and IMDB, where temporal edges contain only one timestamp. The reason is that DBLP only records date of publication of research papers and for each research paper, there is no end time of publication or existence. We only focus on the fact that at a certain time, some researchers have published a certain paper. It's the same with movies in IMDB.
%There are Email communication network is another example. Temporal edges in email communication network contain only timestamp of sending the email, and there is no timestamp of finishing sending the email or timestamp of deleting the email. Sending email is considered as an instant activity.

Our method can be adapted for maintaining temporal edges with time range. The key is to regard each temporal edge as a series of temporal edges containing only one timestamp which is in the original time range. In detail, for a temporal edge $ (u,v,t_s,t_e) $, we can regard $ (u,v,t_s,t_e) $ as $ (t_e-t_s) $ temporal edges with single timestamp $ (u,v,t_s+0.5),(u,v,t_s+1.5),...,(u,v,t_e-0.5) $. Aggregation function on temporal edges COUNT($\ast$) can be re-defined as: asking for the total count of time units in connections between two vertices or two groups of vertices (COUNT($\ast$) can also be regarded as Duration($\ast$) in this case). For a temporal edge $ (u,v,t_s,t_e) $ with time range $ [t_s,t_e] $ and a query $ [t_1,t_2] $ with COUNT($\ast$) satisfying $ t_1<t_s<t_2<t_e $, the aggregation result of $ (u,v) $ should be $ (t_2-t_s) $. We regard $ (u,v,t_s,t_e) $ as temporal edges with single timestamp $ (u,v,t_s+0.5),(u,v,t_s+1.5),...,(u,v,t_e-0.5) $, and only $ (u,v,t_s+0.5),(u,v,t_s+1.5),...,(u,v,t_2-0.5) $ are in range $ [t_1,t_2] $, so there are total $ (t_2-0.5-(t_s+0.5)+1)=(t_2-t_s) $ time units in range $ [t_1,t_2] $, which is consist with the above result. In the practical we can let all timestamps be multiplied by 2. In this case, using query $ [2t_1,2t_2] $ with COUNT($\ast$) we can obtain the same result. Finally we can continue to use STree to accelerate the above query.

Overall, our method can be easily modified to support temporal networks containing temporal edges with time range. We have added the above discussion to the end of Section 7.1 of the revised version.

\begin{Comments}
D2: Following D1, in current experimental datasets, DBLP and IMDB, all edges with single timestamps. I suggest adding experiments with datasets which contain edges with time ranges.
\end{Comments}
\noindent \textbf{Response:} Thanks for your advice. We have further built temporal networks contain edges with time range based on IMDB, since vertices in IMDB has enough number of dimensions. We randomly assign time ranges to each of the original temporal edge. The experimental results of a series of temporal queries are similar to our results before. The reason is that we don't need to modify the index. As we mentioned in the response of D1, we can transform temporal edges with time range into a series of temporal edges with one timestamp. The only difference is that all timestamps are multiplied by 2. So the time complexity of conducting a range query is almost the same. Meanwhile, the theoretical analysis in Section 5 still holds since snapshot is still maintained by node of STree, so similarity of snapshots still affects acceleration ratio and space consumption ratio of the index.
%One possible way is to manually generate suitable datasets. However, we do not have a priori knowledge to guide us in designing time range for temporal edges. Also, as we analyzed in response to D1, only Duration($\ast$) is the reasonable aggregation function on temporal edges with time range, and conducting temporal queries on temporal edges with time range using Duration($\ast$) is equivalent to conducting temporal queries on temporal edges with single timestamp using COUNT($\ast$). The reasons are as follows: for a temporal edges $ (u,v,t_s,t_e) $ with time range $ [t_s,t_e] $ and a query $ [t_1,t_2] $ with Duration($\ast$) satisfying $ t_1<t_s<t_2<t_e $, the aggregation result of $ (u,v) $ is $ (t_2-t_s) $. We can regard $ (u,v,t_s,t_e) $ as $ (t_e-t_s) $ temporal edges with single timestamp $ (u,v,t_s+0.5),(u,v,t_s+1.5),...,(u,v,t_e-0.5) $. In the practical we can let all timestamps be multiplied by 2. In this case, using query $ [2t_1,2t_2] $ with COUNT($\ast$) we can obtain the same result. We don't need to modify the index and there is only slight difference between the time complexity of conducting the above queries.


\begin{Comments}
D3: The effectiveness of partial materialization is not verified in experiments.
\end{Comments}
\noindent \textbf{Response:} Thanks for reminding us. We have added relevant experiments in Exp-3 of Section 7.3. In Exp-3 we use four partial materialization strategies: Random strategy and MinLevel with $ l_0 $ equals $ 2,3,4 $. We have analyzed in Section 6 that the traditional greedy strategy can not be used in partial materialization of temporal graph cube because of the uncertain time range specified by user in queries, which do not exist in traditional OLAP model. Given $ k $, Random strategy selects $ k $ views randomly to be materialized. Since users are more willing to query with less dimensions, we assume that the number of queries with $ x $ dimensions is twice as many as the number of queries with $ x+1 $ dimensions. The results in Figure 13 show that as $ k $ increases, all of the 4 strategies achieve better performance because more views are materialized. Among the 4 strategies, MinLevel with $ l_0=4 $ achieves the best performance because queries specified by users with no more 4 dimensions take up a major part, and these queries are more possible to be accelerated compared to the case when Random strategy is used. As a result, with proper setting of $ l_0 $, MinLevel is the most suitable method for partial materialization of temporal graph cube.

\section{Response to Reviewer \#3}
\setcounter{Comments}{0}
\begin{Comments}
The experimental study only tests the proposed methods in this paper. I know the temporal graph cube is a novel problem studied in this paper, thus there is no existing algorithms/methods can direct handle it. However, is it possible to adapt some existing methods on graph summarizing, or static graph cube methods to handle the temporal graph cube problem as some baselines? Through comparing with modified existing methods can better show the advantages of the proposed methods.
\end{Comments}
\noindent \textbf{Response:} It's indeed possible to adapt existing methods as baselines. We have added experiments in both Section 7.2 and Exp-1 of Section 7.3. First, in Section 7.2 we add experimental results of queries in static graph cube. Our idea is to compare the results obtained from static graph cube and temporal graph cube. By comparing the query results we show that queries in temporal graph cube can reveal hidden information which may be ignored using queries in static graph cube. Second, at the end of Exp-1 of Section 7.3 we implement two other kinds of indexes, prefix array and sparse table, to accelerate range queries of temporal graph cube. The reason is that prefix array and sparse table are both classic methods to accelerate range query in numeric array, and they both can achieve $ O(1) $ time complexity in conducting range queries. However, in the case of snapshot arrays, only sparse table achieves higher acceleration ratio, but with unacceptable space consumption ratio (sometimes out of memory). Above results show the advantage of segment-tree based method in manuscript, which achieves a balance between efficiency and space consumption.


\begin{Comments}
 some minor language issues should be addressed. E.g., In line 52, left column, page 2. "a index" $ \rightarrow $ "an index"
\end{Comments}
\noindent \textbf{Response:} Thank you very much for pointing out the negligence. We have corrected this error.


\end{document}


